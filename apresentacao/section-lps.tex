\section{Linha de produto de software}

\begin{frame}
\frametitle{Linha de produção}

Processo desenvolvido por \textit{Henry Ford}, iniciado no dia 7 de outubro de 1913.

\begin{figure}
\pgfuseimage{fig:productLineHenryFord}
\caption{Linha de produção criada por \textit{Henry Ford}}
\end{figure}

\end{frame}

\begin{frame}
\frametitle{Linha de produção}

Forma de produção em série, onde operários e máquinas realizam tarefas especializadas;

\begin{figure}
\pgfuseimage{fig:lpIphone5Foxconn}
\caption{Linha de produção do Iphone 5 - Foxconn}
\end{figure}

\end{frame}

\begin{frame}
\frametitle{Família de produtos}

% http://www.simplessolucoes.com.br/blog/2011/10/definicao-das-familias-de-produtos-em-ambientes-de-producao-com-baixo-volume-e-alta-variedade-de-produtos-%E2%80%93-um-estudo-de-caso-sergio-ari-tonezer/
\begin{block}{Família de produtos}
  Uma família é um grupo de produtos que passam por etapas semelhantes de processamento e utilizam equipamentos comuns nos processos anteriores. (ROTHER e SHOOK, 2003)
\end{block}

\begin{itemize}
 \item {
   São todos os produtos que podem ser gerados por uma linha de produto específica;
 }
\end{itemize}

\end{frame}

\begin{frame}
\frametitle{Família de produtos}
\framesubtitle{Minions}

\begin{figure}
\pgfuseimage{fig:lp-minions}
\caption{Personagens Minions do filme Meu Malvado Favorito}
\end{figure}

\end{frame}

\begin{frame}
\frametitle{Família de produtos}
\framesubtitle{Subway}

\begin{figure}
\pgfuseimage{fig:lp-subway}
\caption{Família de produtos da rede de franquias Subway}
\end{figure}

\end{frame}

\begin{frame}
\frametitle{Família de produtos}
\framesubtitle{Núcleo de artefatos - Similaridades}

São as características comuns - similaridades - entre os produtos de uma família;

    \centering
    \begin{figure}
    \pgfuseimage{fig:carro1-frente}
    \pgfuseimage{fig:carro2-frente}
    \pgfuseimage{fig:carro3-frente}
    \caption{Similaridade entre os produtos de uma família de produtos}
    \end{figure}

\end{frame}

\begin{frame}
\frametitle{Família de produtos}
\framesubtitle{Variabilidades}

São as características que podem variar entre os produtos de uma família de produtos;

    \centering
    \begin{figure}
    \pgfuseimage{fig:carro1-tras}
    \pgfuseimage{fig:carro2-tras}
    \pgfuseimage{fig:carro3-tras}
    \caption{Variabilidades entre os produtos de uma família de produtos}
    \end{figure}

\end{frame}

\begin{frame}
\frametitle{Linha de produto de software (LPS)}

\begin{block}{Linha de Produto de Software (LPS)}Abordagem que visa a promover a geração de produtos específicos com base na reutilização de uma infraestrutura central - núcleo de artefatos - formada por uma arquitetura de software e seus componentes.\end{block}

\end{frame}

\begin{frame}
\frametitle{Linha de produto de software (LPS)}
\framesubtitle{Vantagens}

\begin{itemize}
 \item {
   Por meio desta abordagem, é possível explorar as semelhanças dos seus produtos para aumentar o \textbf{reuso} de artefatos. 
 }
 \item {
   \textbf{Vantagens:}
   
   \begin{itemize}
     \item {
       Maior reutilização de artefatos;
     }
%      \item {
%        Maximização de lucros;
%      }
     \item {
      Diminuição do \textit{time to market};
    }
    \item {
      Diminuição de riscos;
    }
    \item {
      Produtos com maior qualidade;
    }
    \item {
      ROI;
    }
   \end{itemize}
   
 }
\end{itemize}

\end{frame}

\begin{frame}
\frametitle{Linha de produto de software (LPS)}
\framesubtitle{Atividades essenciais}

  \begin{columns}[onlytextwidth]
  
    \begin{column}{0.5\textwidth}
      \centering
      \begin{figure}
      \pgfuseimage{fig:atividadesEssenciaisLps}
      % \caption{Linha de produção do Iphone 5 - Foxconn}
      \end{figure}
    \end{column}
    
    \begin{column}{0.5\textwidth}
      \begin{itemize}
        \item {
          \textbf{Desen. Núcleo de Artefatos} esta relacionado com a Arquitetura da LPS;
        }
        \item {
          \textbf{Desen. dos Produtos} é responsável pela geração dos produtos específicos da LPS;
        }
      \end{itemize}

    \end{column}
    
  \end{columns}

\end{frame}

\begin{frame}
\frametitle{Linha de produto de software (LPS)}
\framesubtitle{Definições}

  \begin{itemize}
    \item {
      A base de uma LPS é o seu \textbf{núcleo de artefatos};
    }
    \item {
      Uma \textbf{característica} (\textit{feature}) é uma capacidade do sistema que é relevante e visível para o usuário final;
    }
    \item {
      A definição explícita de variabilidades em LPS é a \textbf{diferença} chave entre o desenvolvimento de \textbf{sistemas únicos} e o \textbf{desenvolvimento de LPS}.
    }
  \end{itemize}

\end{frame}

\begin{frame}
\frametitle{Linha de produto de software (LPS)}
\framesubtitle{Definições}

  \begin{itemize}
    \item {
      \textbf{Ponto de variação}: Um \textbf{local específico} de um artefato em que uma decisão de projeto ainda não foi tomada;
    }
    \item {
      \textbf{Variante}: Corresponde a uma \textbf{alternativa de projeto} para resolver uma determinada variabilidade;
    }
    \item {
      \textbf{Restrições entre variantes}: define os relacionamentos entre duas ou mais variantes para que seja possível resolver um ponto de variação ou uma variabilidade.
    }
  \end{itemize}

\end{frame}

\begin{frame}
\frametitle{Artefatos de uma Linha de Produtos}

  \begin{figure}
  \pgfuseimage{fig:artefatos1}
  \caption{Linha de produto}
  \end{figure}

\end{frame}

\begin{frame}
\frametitle{Artefatos de uma Linha de Produtos}
\framesubtitle{Pontos de variação}

  \begin{figure}
  \pgfuseimage{fig:artefatos2}
  \caption{Linha de produto}
  \end{figure}

\end{frame}

\begin{frame}
\frametitle{Artefatos de uma Linha de Produtos}
\framesubtitle{Pontos de variação}

  \begin{figure}
  \pgfuseimage{fig:artefatos3}
  \caption{Linha de produto}
  \end{figure}

\end{frame}

\begin{frame}
\frametitle{Artefatos de uma Linha de Produtos}
\framesubtitle{Pontos de variação}

  \begin{figure}
  \pgfuseimage{fig:artefatos4}
  \caption{Linha de produto}
  \end{figure}

\end{frame}

\begin{frame}
\frametitle{Artefatos de uma Linha de Produtos}
\framesubtitle{Pontos de variação}

  \begin{figure}
  \pgfuseimage{fig:artefatos5}
  \caption{Linha de produto}
  \end{figure}

\end{frame}

\begin{frame}
\frametitle{Artefatos de uma Linha de Produtos}
\framesubtitle{Similaridades}

  \begin{figure}
  \pgfuseimage{fig:artefatos6}
  \caption{Linha de produto}
  \end{figure}

\end{frame}

\begin{frame}
\frametitle{Artefatos de uma Linha de Produtos}
\framesubtitle{Similaridades}

  \begin{figure}
  \pgfuseimage{fig:artefatos7}
  \caption{Linha de produto}
  \end{figure}

\end{frame}

\begin{frame}
\frametitle{Artefatos de uma Linha de Produtos}
\framesubtitle{Pontos de variação}

  \begin{figure}
  \pgfuseimage{fig:artefatos8}
  \caption{Linha de produto}
  \end{figure}

\end{frame}
