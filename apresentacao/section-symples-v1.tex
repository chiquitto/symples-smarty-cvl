\section{Abordagem SyMPLES v1.0}

\begin{frame}
\frametitle{SyMPLES}

\begin{block}{SyMPLES}
   Abordagem de Desenvolvimento de Linha de Produto para Sistemas Embarcados baseada em \textit{SysML};
\end{block}

\end{frame}


\begin{frame}
\frametitle{SyMPLES}

\begin{itemize}
 \item {
   Combina modelos de alto nível e técnicas de LP para o desenvolvimento de Sistemas Embarcados;
 }
 \item {
   Utiliza a linguagem \textit{SysML} (extensão da linguagem \textit{UML}) como base para representação dos modelos LPS;
 }
 
\end{itemize}

\end{frame}

\begin{frame}
\frametitle{Abordagem SMarty}

\begin{block}{SMarty}
   Permite o gerenciamento de variabilidades em uma LPS modeladas em \textit{UML}/\textit{SysML}.
\end{block}

\textit{SyMPLES v1.0} utiliza a abordagem \textit{SMarty} para o gerenciamento de variabilidades. A abordagem \textit{SMarty} pode ser aplicada tanto para modelos \textit{UML} tanto para modelos \textit{SysML}.

\end{frame}


\begin{frame}
\frametitle{SyMPLES v1.0}
\framesubtitle{Perfis e Processos}

A abordagem \textit{SyMPLES v1.0} é composta de:

\begin{itemize}
 \item {
   \textit{SyMPLES-ProfileFB};
 }
 \item {
   \textit{SyMPLES-ProfileVar};
 }
 \item {
   \textit{SyMPLES-ProcessPL};
 }
 \item {
   \textit{SyMPLES-ProcessVar};
 }
 
\end{itemize}

\end{frame}


\begin{frame}
\frametitle{SyMPLES v1.0}
\framesubtitle{Perfil SyMPLES-ProfileVar}

\begin{itemize}
 \item {
   Baseado no \textit{SMartyProfile} da abordagem \textit{SMarty};
 }
 \item {
   Conjunto de estereótipos para expressar variabilidade;
 }
\end{itemize}

\end{frame}


\begin{frame}
\frametitle{SyMPLES v1.0}
\framesubtitle{Estereótipos de SyMPLES-ProfileVar}

\begin{columns}[onlytextwidth]
  
    \begin{column}{0.5\textwidth}
      \begin{itemize}
        \item {
          \texttt{<<variationPoint>>}
        }
        \item {
          \texttt{<<variability>>}
        }
        \item {
          \texttt{<<requires>>}
        }
        \item {
          \texttt{<<mutex>>}
        }
      \end{itemize}
    \end{column}
    
    \begin{column}{0.5\textwidth}
      \begin{itemize}
        \item {
          \texttt{<<variant>>}
          
          \begin{itemize}
            \item {
              \texttt{<<mandatory>>}
            }
            \item {
              \texttt{<<optional>>}
            }
            \item {
              \texttt{<<alternative\_OR>>}
            }
            \item {
              \texttt{<<alternative\_XOR>>}
            }
          \end{itemize}
          
        }
      \end{itemize}
    \end{column}
    
  \end{columns}

\end{frame}

\subsection{Exemplo}

\begin{frame}
\frametitle{SyMPLES v1.0}
\framesubtitle{LPS de uma impressora - Diagrama de Definição de Blocos}

  \begin{figure}
  \pgfuseimage{fig:printer2SmartyBlocks}
  \end{figure}

\end{frame}


\begin{frame}
\frametitle{SyMPLES v1.0}

  \texttt{<<variationPoint>>}: representa o local em que ocorre uma variabilidade. Um ponto de variação está sempre associado à uma ou mais variantes.

\end{frame}


\begin{frame}
\frametitle{SyMPLES v1.0}

  \texttt{<<mandatory>>}: a variante deve estar obrigatoriamente presente na configuração de qualquer modelo de LPS.

\end{frame}


\begin{frame}
\frametitle{SyMPLES v1.0}
\framesubtitle{Exemplo de uso}

  \begin{columns}[onlytextwidth]
  
    \begin{column}{0.5\textwidth}
      \begin{figure}
      \pgfuseimage{fig:printer2SmartyBlocksPedacoMandatory}
      \end{figure}
    \end{column}
    
    \begin{column}{0.5\textwidth}
      \textbf{Exemplo}: o bloco \texttt{Tonner} deve obrigatoriamente estar em todos os produtos gerados pela LPS, por isso ele foi marcado com o estereótipo <<mandatory>>
    \end{column}
    
  \end{columns}

\end{frame}


\begin{frame}
\frametitle{SyMPLES v1.0}

  \texttt{<<optional>>}: a variante deve estar obrigatoriamente presente na configuração de qualquer modelo de LPS.

\end{frame}


\begin{frame}
\frametitle{SyMPLES v1.0}
\framesubtitle{Exemplo de uso}

  \begin{columns}[onlytextwidth]
    
    \begin{column}{0.6\textwidth}
      \textbf{Exemplo}: o bloco \texttt{EmgPower} foi marcado com o estereótipo \texttt{<<optional>>} pois ele é uma caracteristica opcional nos produtos gerados pela LPS.
    \end{column}
  
    \begin{column}{0.4\textwidth}
      \begin{figure}
      \pgfuseimage{fig:printer2SmartyBlocksPedacoOptional}
      \end{figure}
    \end{column}
    
  \end{columns}

\end{frame}


\begin{frame}
\frametitle{SyMPLES v1.0}

  \texttt{<<alternative\_OR>>}: estão sempre associadas aos pontos de variação. \textbf{Pelo menos uma} das variantes deverá ser escolhida para resolver o ponto de variação, ou seja, para estar presente na configuração da LPS.

\end{frame}


\begin{frame}
\frametitle{SyMPLES v1.0}
\framesubtitle{Exemplo de uso}

  \begin{columns}[onlytextwidth]
  
    \begin{column}{0.4\textwidth}
      \begin{figure}
      \pgfuseimage{fig:printer2SmartyBlocksPedacoAltOr}
      \end{figure}
    \end{column}
    
    \begin{column}{0.6\textwidth}
      \textbf{Exemplo}: o blocos \texttt{Usb2Connector}, \texttt{Usb3Connector} e \texttt{WifiConnector} foram marcados com o estereótipo \texttt{<<alternative\_OR>>} pois são características \textbf{alternativas inclusivas}, ou seja, pelo menos uma das variantes deve existir no produto gerado pela LPS.
    \end{column}
    
  \end{columns}

\end{frame}


\begin{frame}
\frametitle{SyMPLES v1.0}

  \texttt{<<alternative\_XOR>>}: estão sempre associadas aos pontos de variação. \textbf{Somente uma} das variantes deverá ser escolhida para resolver o ponto de variação.

\end{frame}


\begin{frame}
\frametitle{SyMPLES v1.0}
\framesubtitle{Exemplo de uso}

  \begin{columns}[onlytextwidth]
    
    \begin{column}{0.6\textwidth}
      \textbf{Exemplo}: o blocos \texttt{BWTonner} e \texttt{ColorTonner} foram marcados com o estereótipo \texttt{<<alternative\_XOR>>} pois são características \textbf{alternativas exclusivas}, ou seja, somente um bloco entre \texttt{BWTonner} e \texttt{ColorTonner} deve existir no produto gerado pela LPS.
    \end{column}
  
    \begin{column}{0.4\textwidth}
      \begin{figure}
      \pgfuseimage{fig:printer2SmartyBlocksPedacoAltXor}
      \end{figure}
    \end{column}
    
  \end{columns}

\end{frame}

\begin{frame}
\frametitle{SyMPLES v1.0}

  \texttt{<<variability>>}: indica uma variabilidade existente em um modelo UML.

\end{frame}


\begin{frame}
\frametitle{SyMPLES v1.0}
\framesubtitle{Exemplo de uso}

  \textbf{Exemplo}: o bloco \texttt{EmgPower} foi relacionado com uma nota estereotipada com \texttt{<<variability>>}, pois \texttt{EmgPower} representa um ponto de variação;

  \begin{figure}
  \pgfuseimage{fig:printer2SmartyBlocksPedacoVariability}
  \end{figure}

\end{frame}


\begin{frame}
\frametitle{SyMPLES v1.0}

  \texttt{<<requires>>}: indica um relacionamento de dependência (em \textit{UML}/\textit{SysML}) entre variantes no qual a
variante dependente (origem da dependência) só existirá em uma configuração se a variante relacionada (destino da dependência) existir.

\end{frame}


\begin{frame}
\frametitle{SyMPLES v1.0}
\framesubtitle{Exemplo de uso}
  
  \begin{columns}[onlytextwidth]
  
    \begin{column}{0.4\textwidth}
      \begin{figure}
      \pgfuseimage{fig:printer2SmartyBlocksPedacoRequires}
      \end{figure}
    \end{column}
    
    \begin{column}{0.6\textwidth}
      \textbf{Exemplo}: o bloco \texttt{ColorTonner} requer a existência do bloco \texttt{Usb3Connector}, então estes dois blocos foram relacionados com uma dependência estereotipada com \texttt{<<requires>>};
    \end{column}
    
  \end{columns}

\end{frame}


\begin{frame}
\frametitle{SyMPLES v1.0}

  \texttt{<<mutex>>}: indica um relacionamento de dependência (em \textit{UML}/\textit{SysML}) entre variantes no qual a variante dependente (origem da dependência) só existirá em uma configuração se a variante relacionada (destino da dependência) obrigatoriamente não existir.

  \begin{itemize}
    \item {
      São conhecidas como variantes mutuamente exclusivas.
    }
  \end{itemize}

\end{frame}


\begin{frame}
\frametitle{SyMPLES v1.0}
\framesubtitle{Exemplo de uso}
  
  \begin{columns}[onlytextwidth]
    
    \begin{column}{0.4\textwidth}
      \textbf{Exemplo}: o bloco \texttt{EmgPower} requer a exclusão do bloco \texttt{WifiConnector};
    \end{column}
  
    \begin{column}{0.6\textwidth}
      \begin{figure}
      \pgfuseimage{fig:printer2SmartyBlocksPedacoMutex}
      \end{figure}
    \end{column}
    
  \end{columns}

\end{frame}
