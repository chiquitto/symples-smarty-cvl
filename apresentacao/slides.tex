% \documentclass{beamer}
% \documentclass[aspectratio=32,draft]{beamer}
\documentclass[aspectratio=32]{beamer}

%%%%%%%%% ALISSON CHIQUITTO

\usepackage[brazil]{babel}
\usepackage[utf8]{inputenc}
%\usepackage[T1]{fontenc}
%\usepackage[hyphens]{url}
%\usepackage{graphicx}
%\usepackage[pdftex]{graphicx}
%\usepackage{epstopdf}
% \usepackage{cite}
%\usepackage{color}
\usepackage{enumerate}
%\usepackage{caption}
%\usepackage{hyperref}

%\usepackage{ulem} % texto tachado - use \sout{texto}
\usepackage[normalem]{ulem} % Fazer sublinhado e tachado em textos \uline{sublinhado} e \sout{tachado}

\usepackage{xcolor}
\definecolor{blue}{rgb}{0.19, 0.19, 0.7}
% \textcolor{blue}{texto}

% \definecolor{bluelink}{rgb}{0.19, 0.19, 0.7}
% \hypersetup{colorlinks=false,linkcolor=blue,urlcolor=blue}
% \hypersetup{%
%   colorlinks=true,% hyperlinks will be black
%   linkbordercolor=bluelink,% hyperlink borders will be red
%   linkcolor=blue,
%   pdfborderstyle={/S/U/W 1}% border style will be underline of width 1pt
% }

% Definir o caminho das figuras
%\DeclareGraphicsExtensions{.png,.pdf}
%\graphicspath{{img/}}

%%%%%%%%% /ALISSON CHIQUITTO

% There are many different themes available for Beamer. A comprehensive
% list with examples is given here:
% http://deic.uab.es/~iblanes/beamer_gallery/index_by_theme.html
% You can uncomment the themes below if you would like to use a different
% one:
%\usetheme{AnnArbor}
%\usetheme{Antibes} % Section and subsection at top
%\usetheme{Bergen}
%\usetheme{Berkeley}
%\usetheme{Berlin}
\usetheme{Boadilla} % -----
%\usetheme{boxes}
%\usetheme{CambridgeUS}
%\usetheme{Copenhagen}
%\usetheme{Darmstadt}
%\usetheme{default} % ok
%\usetheme{Frankfurt}
%\usetheme{Goettingen}
%\usetheme{Hannover}
%\usetheme{Ilmenau}
%\usetheme{JuanLesPins}
%\usetheme{Luebeck}
%\usetheme{Madrid} % Padrao
%\usetheme{Malmoe}
%\usetheme{Marburg}
%\usetheme{Montpellier}
%\usetheme{PaloAlto}
% \usetheme{Pittsburgh} % ok
%\usetheme{Rochester}
%\usetheme{Singapore}
%\usetheme{Szeged}
%\usetheme{Warsaw}

\title[Experimento - Abordagem SyMPLES]{Gerenciamento de Variabilidade em Sistemas Embarcados através da abordagem SyMPLES}

% A subtitle is optional and this may be deleted
% \subtitle{O básico}

% \author{F.~Author\inst{1} \and S.~Another\inst{2}}
% \author[Alisson Chiquitto]{Discente: Alisson Chiquitto\inst{2} \and Doscente: Prof. Dra. Itana M. S. Gimemes\inst{2}}
% - Give the names in the same order as the appear in the paper.
% - Use the \inst{?} command only if the authors have different
%   affiliation.

% \institute[Universities of Somewhere and Elsewhere] % (optional, but mostly needed)
% {
%   \inst{1}%
%   Department of Computer Science\\
%   University of Somewhere
%   \and
%   \inst{2}%
%   Department of Theoretical Philosophy\\
%   University of Elsewhere}
% - Use the \inst command only if there are several affiliations.
% - Keep it simple, no one is interested in your street address.

\date{Maringá, 2014}
% - Either use conference name or its abbreviation.
% - Not really informative to the audience, more for people (including
%   yourself) who are reading the slides online

%\subject{Theoretical Computer Science}
% This is only inserted into the PDF information catalog. Can be left
% out. 

% If you have a file called "university-logo-filename.xxx", where xxx
% is a graphic format that can be processed by latex or pdflatex,
% resp., then you can add a logo as follows:

% \pgfdeclareimage[height=0.5cm]{university-logo}{university-logo-filename}
% \logo{\pgfuseimage{university-logo}}

% Delete this, if you do not want the table of contents to pop up at
% the beginning of each subsection:
\AtBeginSection[]
{
  \begin{frame}<beamer>{Outline}
    \tableofcontents[currentsection]
  \end{frame}
}

\AtBeginSubsection[]
{
  \begin{frame}<beamer>{Outline}
    \tableofcontents[currentsection,currentsubsection]
  \end{frame}
}

% Adapted from beamerinnerthemedfault.sty
% http://tex.stackexchange.com/questions/11168/change-bullet-style-formatting-in-beamer
% \setbeamertemplate{itemize item}{\scriptsize\raise1.25pt\hbox{\donotcoloroutermaths$\blacktriangleright$}}
% \setbeamertemplate{itemize subitem}{\tiny\raise1.5pt\hbox{\donotcoloroutermaths$\blacktriangleright$}}
% \setbeamertemplate{itemize subsubitem}{\tiny\raise1.5pt\hbox{\donotcoloroutermaths$\blacktriangleright$}}
\setbeamertemplate{enumerate item}{\insertenumlabel.}
\setbeamertemplate{enumerate subitem}{\insertenumlabel.\insertsubenumlabel}
\setbeamertemplate{enumerate subsubitem}{\insertenumlabel.\insertsubenumlabel.\insertsubsubenumlabel}
\setbeamertemplate{enumerate mini template}{\insertenumlabel}

\setbeamertemplate{frametitle continuation}[from second]

% \pgfdeclareimage[width=1.65in]{fig:hello-world}{img/hello-world.jpg}
% \begin{figure}
% \pgfuseimage{fig:html-form-fieldset}
% \caption{Um grupo de elementos}
% \end{figure}

% 100px = 0.55in - 150px = 0.825in
% 200px = 1.1in - 250px = 1.37in - 275px = 1.51in
% 300px = 1.65in - 350px = 1.92in - 380px = xin
% 400px = 2.2in - 450px = 2.47in
% 500px = 2.75in - 550px = 3.025in
% 600px = 3.3in
% 700px = 3.85in
% 800px = 4.4in
\pgfdeclareimage[width=3.3in]{fig:productLineHenryFord}{img/irr2BgmCotrI-editado.jpg}
\pgfdeclareimage[width=3.3in]{fig:lpIphone5Foxconn}{img/lp-iphone5-foxconn.jpg}
\pgfdeclareimage[width=2.2in]{fig:atividadesEssenciaisLps}{img/atividades-essenciais-lps.jpg}

\pgfdeclareimage[width=3.85in]{fig:lp-minions}{img/lp-minions.png}
\pgfdeclareimage[width=3.3in]{fig:lp-subway}{img/lp-subway.png}

\pgfdeclareimage[width=1.55in]{fig:carro1-frente}{img/carro1-frente.jpg}
\pgfdeclareimage[width=1.55in]{fig:carro1-tras}{img/carro1-tras.jpg}
\pgfdeclareimage[width=1.55in]{fig:carro2-frente}{img/carro2-frente.jpg}
\pgfdeclareimage[width=1.55in]{fig:carro2-tras}{img/carro2-tras.jpg}
\pgfdeclareimage[width=1.55in]{fig:carro3-frente}{img/carro3-frente.jpg}
\pgfdeclareimage[width=1.55in]{fig:carro3-tras}{img/carro3-tras.jpg}

\pgfdeclareimage[width=3.85in]{fig:artefatos1}{img/artefatos1.png}
\pgfdeclareimage[width=3.85in]{fig:artefatos2}{img/artefatos2.png}
\pgfdeclareimage[width=3.85in]{fig:artefatos3}{img/artefatos3.png}
\pgfdeclareimage[width=3.85in]{fig:artefatos4}{img/artefatos4.png}
\pgfdeclareimage[width=3.85in]{fig:artefatos5}{img/artefatos5.png}
\pgfdeclareimage[width=3.85in]{fig:artefatos6}{img/artefatos6.png}
\pgfdeclareimage[width=3.85in]{fig:artefatos7}{img/artefatos7.png}
\pgfdeclareimage[width=3.85in]{fig:artefatos8}{img/artefatos8.png}

\pgfdeclareimage[width=2.75in]{fig:printer2SmartyBlocks}{img/printer2-smarty-blocks.pdf}
\pgfdeclareimage[width=2.2in]{fig:printer2SmartyBlocksPedacoMandatory}{img/printer2-smarty-blocks-pedaco-mandatory.pdf}
\pgfdeclareimage[width=2.2in]{fig:printer2SmartyBlocksPedacoOptional}{img/printer2-smarty-blocks-pedaco-optional.pdf}
\pgfdeclareimage[width=1.65in]{fig:printer2SmartyBlocksPedacoAltOr}{img/printer2-smarty-blocks-pedaco-altOr.pdf}
\pgfdeclareimage[width=1.65in]{fig:printer2SmartyBlocksPedacoAltXor}{img/printer2-smarty-blocks-pedaco-altXor.pdf}
\pgfdeclareimage[width=2.75in]{fig:printer2SmartyBlocksPedacoVariability}{img/printer2-smarty-blocks-pedaco-variability.pdf}
\pgfdeclareimage[width=1.65in]{fig:printer2SmartyBlocksPedacoMutex}{img/printer2-smarty-blocks-pedaco-mutex.pdf}
\pgfdeclareimage[width=1.65in]{fig:printer2SmartyBlocksPedacoRequires}{img/printer2-smarty-blocks-pedaco-requires.pdf}

\pgfdeclareimage[width=2.75in]{fig:printer2CvlBlocksImplies}{img/printer2-cvl-blocks-implies.pdf}
\pgfdeclareimage[width=1.65in]{fig:printer2CvlBlocksIsImpliedByParentF}{img/printer2-cvl-blocks-isImpliedByParentF.pdf}
\pgfdeclareimage[width=1.1in]{fig:printer2CvlBlocksIsImpliedByParentT}{img/printer2-cvl-blocks-isImpliedByParentT.pdf}
\pgfdeclareimage[width=2.75in]{fig:printer2CvlBlocksMultiplicidadeGrupo}{img/printer2-cvl-blocks-multiplicidadeGrupo.pdf}

% Let's get started
\begin{document}

\institute[UEM] % (optional, but mostly needed)
{
  Programa de Mestrado em Ciência da Computação
}
\author[Alisson G. Chiquitto]{\begin{tabular}{r@{ }l} 
Discente:       & Alisson Gaspar Chiquitto \\[1ex] 
Orientadora:    & Prof. Dra. Itana M. S. Gimenes
\end{tabular}}

\begin{frame}
  \titlepage
\end{frame}

\begin{frame}{Sumário}
  \tableofcontents
  % You might wish to add the option [pausesections]
\end{frame}

% =======

\begin{frame}
\frametitle{Agradecimentos}

  Adaptação do material gentilmente cedido por:
  
  \begin{itemize}
    \item {
      Anderson da Silva Marcolino;
    }
    \item {
      Marcio Bera; e
    }
    \item {
      Ricardo Theis Geraldi.
    }
  \end{itemize}

\end{frame}

\section{Linha de produto de software}

\begin{frame}
\frametitle{Linha de produção}

Processo desenvolvido por \textit{Henry Ford}, iniciado no dia 7 de outubro de 1913.

\begin{figure}
\pgfuseimage{fig:productLineHenryFord}
\caption{Linha de produção criada por \textit{Henry Ford}}
\end{figure}

\end{frame}

\begin{frame}
\frametitle{Linha de produção}

Forma de produção em série, onde operários e máquinas realizam tarefas especializadas;

\begin{figure}
\pgfuseimage{fig:lpIphone5Foxconn}
\caption{Linha de produção do Iphone 5 - Foxconn}
\end{figure}

\end{frame}

\begin{frame}
\frametitle{Família de produtos}

% http://www.simplessolucoes.com.br/blog/2011/10/definicao-das-familias-de-produtos-em-ambientes-de-producao-com-baixo-volume-e-alta-variedade-de-produtos-%E2%80%93-um-estudo-de-caso-sergio-ari-tonezer/
\begin{block}{Família de produtos}
  Uma família é um grupo de produtos que passam por etapas semelhantes de processamento e utilizam equipamentos comuns nos processos anteriores. (ROTHER e SHOOK, 2003)
\end{block}

\begin{itemize}
 \item {
   São todos os produtos que podem ser gerados por uma linha de produto específica;
 }
\end{itemize}

\end{frame}

\begin{frame}
\frametitle{Família de produtos}
\framesubtitle{Minions}

\begin{figure}
\pgfuseimage{fig:lp-minions}
\caption{Personagens Minions do filme Meu Malvado Favorito}
\end{figure}

\end{frame}

\begin{frame}
\frametitle{Família de produtos}
\framesubtitle{Subway}

\begin{figure}
\pgfuseimage{fig:lp-subway}
\caption{Família de produtos da rede de franquias Subway}
\end{figure}

\end{frame}

\begin{frame}
\frametitle{Família de produtos}
\framesubtitle{Núcleo de artefatos - Similaridades}

São as características comuns - similaridades - entre os produtos de uma família;

    \centering
    \begin{figure}
    \pgfuseimage{fig:carro1-frente}
    \pgfuseimage{fig:carro2-frente}
    \pgfuseimage{fig:carro3-frente}
    \caption{Similaridade entre os produtos de uma família de produtos}
    \end{figure}

\end{frame}

\begin{frame}
\frametitle{Família de produtos}
\framesubtitle{Variabilidades}

São as características que podem variar entre os produtos de uma família de produtos;

    \centering
    \begin{figure}
    \pgfuseimage{fig:carro1-tras}
    \pgfuseimage{fig:carro2-tras}
    \pgfuseimage{fig:carro3-tras}
    \caption{Variabilidades entre os produtos de uma família de produtos}
    \end{figure}

\end{frame}

\begin{frame}
\frametitle{Linha de produto de software (LPS)}

\begin{block}{Linha de Produto de Software (LPS)}Abordagem que visa a promover a geração de produtos específicos com base na reutilização de uma infraestrutura central - núcleo de artefatos - formada por uma arquitetura de software e seus componentes.\end{block}

\end{frame}

\begin{frame}
\frametitle{Linha de produto de software (LPS)}
\framesubtitle{Vantagens}

\begin{itemize}
 \item {
   Por meio desta abordagem, é possível explorar as semelhanças dos seus produtos para aumentar o \textbf{reuso} de artefatos. 
 }
 \item {
   \textbf{Vantagens:}
   
   \begin{itemize}
     \item {
       Maior reutilização de artefatos;
     }
%      \item {
%        Maximização de lucros;
%      }
     \item {
      Diminuição do \textit{time to market};
    }
    \item {
      Diminuição de riscos;
    }
    \item {
      Produtos com maior qualidade;
    }
    \item {
      ROI;
    }
   \end{itemize}
   
 }
\end{itemize}

\end{frame}

\begin{frame}
\frametitle{Linha de produto de software (LPS)}
\framesubtitle{Atividades essenciais}

  \begin{columns}[onlytextwidth]
  
    \begin{column}{0.5\textwidth}
      \centering
      \begin{figure}
      \pgfuseimage{fig:atividadesEssenciaisLps}
      % \caption{Linha de produção do Iphone 5 - Foxconn}
      \end{figure}
    \end{column}
    
    \begin{column}{0.5\textwidth}
      \begin{itemize}
        \item {
          \textbf{Desen. Núcleo de Artefatos} esta relacionado com a Arquitetura da LPS;
        }
        \item {
          \textbf{Desen. dos Produtos} é responsável pela geração dos produtos específicos da LPS;
        }
      \end{itemize}

    \end{column}
    
  \end{columns}

\end{frame}

\begin{frame}
\frametitle{Linha de produto de software (LPS)}
\framesubtitle{Definições}

  \begin{itemize}
    \item {
      A base de uma LPS é o seu \textbf{núcleo de artefatos};
    }
    \item {
      Uma \textbf{característica} (\textit{feature}) é uma capacidade do sistema que é relevante e visível para o usuário final;
    }
    \item {
      A definição explícita de variabilidades em LPS é a \textbf{diferença} chave entre o desenvolvimento de \textbf{sistemas únicos} e o \textbf{desenvolvimento de LPS}.
    }
  \end{itemize}

\end{frame}

\begin{frame}
\frametitle{Linha de produto de software (LPS)}
\framesubtitle{Definições}

  \begin{itemize}
    \item {
      \textbf{Ponto de variação}: Um \textbf{local específico} de um artefato em que uma decisão de projeto ainda não foi tomada;
    }
    \item {
      \textbf{Variante}: Corresponde a uma \textbf{alternativa de projeto} para resolver uma determinada variabilidade;
    }
    \item {
      \textbf{Restrições entre variantes}: define os relacionamentos entre duas ou mais variantes para que seja possível resolver um ponto de variação ou uma variabilidade.
    }
  \end{itemize}

\end{frame}

\begin{frame}
\frametitle{Artefatos de uma Linha de Produtos}

  \begin{figure}
  \pgfuseimage{fig:artefatos1}
  \caption{Linha de produto}
  \end{figure}

\end{frame}

\begin{frame}
\frametitle{Artefatos de uma Linha de Produtos}
\framesubtitle{Pontos de variação}

  \begin{figure}
  \pgfuseimage{fig:artefatos2}
  \caption{Linha de produto}
  \end{figure}

\end{frame}

\begin{frame}
\frametitle{Artefatos de uma Linha de Produtos}
\framesubtitle{Pontos de variação}

  \begin{figure}
  \pgfuseimage{fig:artefatos3}
  \caption{Linha de produto}
  \end{figure}

\end{frame}

\begin{frame}
\frametitle{Artefatos de uma Linha de Produtos}
\framesubtitle{Pontos de variação}

  \begin{figure}
  \pgfuseimage{fig:artefatos4}
  \caption{Linha de produto}
  \end{figure}

\end{frame}

\begin{frame}
\frametitle{Artefatos de uma Linha de Produtos}
\framesubtitle{Pontos de variação}

  \begin{figure}
  \pgfuseimage{fig:artefatos5}
  \caption{Linha de produto}
  \end{figure}

\end{frame}

\begin{frame}
\frametitle{Artefatos de uma Linha de Produtos}
\framesubtitle{Similaridades}

  \begin{figure}
  \pgfuseimage{fig:artefatos6}
  \caption{Linha de produto}
  \end{figure}

\end{frame}

\begin{frame}
\frametitle{Artefatos de uma Linha de Produtos}
\framesubtitle{Similaridades}

  \begin{figure}
  \pgfuseimage{fig:artefatos7}
  \caption{Linha de produto}
  \end{figure}

\end{frame}

\begin{frame}
\frametitle{Artefatos de uma Linha de Produtos}
\framesubtitle{Pontos de variação}

  \begin{figure}
  \pgfuseimage{fig:artefatos8}
  \caption{Linha de produto}
  \end{figure}

\end{frame}


% SyMPLES v1.0
\section{Abordagem SyMPLES v1.0}

\begin{frame}
\frametitle{SyMPLES}

\begin{block}{SyMPLES}
   Abordagem de Desenvolvimento de Linha de Produto para Sistemas Embarcados baseada em \textit{SysML};
\end{block}

\end{frame}


\begin{frame}
\frametitle{SyMPLES}

\begin{itemize}
 \item {
   Combina modelos de alto nível e técnicas de LP para o desenvolvimento de Sistemas Embarcados;
 }
 \item {
   Utiliza a linguagem \textit{SysML} (extensão da linguagem \textit{UML}) como base para representação dos modelos LPS;
 }
 
\end{itemize}

\end{frame}

\begin{frame}
\frametitle{Abordagem SMarty}

\begin{block}{SMarty}
   Permite o gerenciamento de variabilidades em uma LPS modeladas em \textit{UML}/\textit{SysML}.
\end{block}

\textit{SyMPLES v1.0} utiliza a abordagem \textit{SMarty} para o gerenciamento de variabilidades. A abordagem \textit{SMarty} pode ser aplicada tanto para modelos \textit{UML} tanto para modelos \textit{SysML}.

\end{frame}


\begin{frame}
\frametitle{SyMPLES v1.0}
\framesubtitle{Perfis e Processos}

A abordagem \textit{SyMPLES v1.0} é composta de:

\begin{itemize}
 \item {
   \textit{SyMPLES-ProfileFB};
 }
 \item {
   \textit{SyMPLES-ProfileVar};
 }
 \item {
   \textit{SyMPLES-ProcessPL};
 }
 \item {
   \textit{SyMPLES-ProcessVar};
 }
 
\end{itemize}

\end{frame}


\begin{frame}
\frametitle{SyMPLES v1.0}
\framesubtitle{Perfil SyMPLES-ProfileVar}

\begin{itemize}
 \item {
   Baseado no \textit{SMartyProfile} da abordagem \textit{SMarty};
 }
 \item {
   Conjunto de estereótipos para expressar variabilidade;
 }
\end{itemize}

\end{frame}


\begin{frame}
\frametitle{SyMPLES v1.0}
\framesubtitle{Estereótipos de SyMPLES-ProfileVar}

\begin{columns}[onlytextwidth]
  
    \begin{column}{0.5\textwidth}
      \begin{itemize}
        \item {
          \texttt{<<variationPoint>>}
        }
        \item {
          \texttt{<<variability>>}
        }
        \item {
          \texttt{<<requires>>}
        }
        \item {
          \texttt{<<mutex>>}
        }
      \end{itemize}
    \end{column}
    
    \begin{column}{0.5\textwidth}
      \begin{itemize}
        \item {
          \texttt{<<variant>>}
          
          \begin{itemize}
            \item {
              \texttt{<<mandatory>>}
            }
            \item {
              \texttt{<<optional>>}
            }
            \item {
              \texttt{<<alternative\_OR>>}
            }
            \item {
              \texttt{<<alternative\_XOR>>}
            }
          \end{itemize}
          
        }
      \end{itemize}
    \end{column}
    
  \end{columns}

\end{frame}

\subsection{Exemplo}

\begin{frame}
\frametitle{SyMPLES v1.0}
\framesubtitle{LPS de uma impressora - Diagrama de Definição de Blocos}

  \begin{figure}
  \pgfuseimage{fig:printer2SmartyBlocks}
  \end{figure}

\end{frame}


\begin{frame}
\frametitle{SyMPLES v1.0}

  \texttt{<<variationPoint>>}: representa o local em que ocorre uma variabilidade. Um ponto de variação está sempre associado à uma ou mais variantes.

\end{frame}


\begin{frame}
\frametitle{SyMPLES v1.0}

  \texttt{<<mandatory>>}: a variante deve estar obrigatoriamente presente na configuração de qualquer modelo de LPS.

\end{frame}


\begin{frame}
\frametitle{SyMPLES v1.0}
\framesubtitle{Exemplo de uso}

  \begin{columns}[onlytextwidth]
  
    \begin{column}{0.5\textwidth}
      \begin{figure}
      \pgfuseimage{fig:printer2SmartyBlocksPedacoMandatory}
      \end{figure}
    \end{column}
    
    \begin{column}{0.5\textwidth}
      \textbf{Exemplo}: o bloco \texttt{Tonner} deve obrigatoriamente estar em todos os produtos gerados pela LPS, por isso ele foi marcado com o estereótipo <<mandatory>>
    \end{column}
    
  \end{columns}

\end{frame}


\begin{frame}
\frametitle{SyMPLES v1.0}

  \texttt{<<optional>>}: a variante deve estar obrigatoriamente presente na configuração de qualquer modelo de LPS.

\end{frame}


\begin{frame}
\frametitle{SyMPLES v1.0}
\framesubtitle{Exemplo de uso}

  \begin{columns}[onlytextwidth]
    
    \begin{column}{0.6\textwidth}
      \textbf{Exemplo}: o bloco \texttt{EmgPower} foi marcado com o estereótipo \texttt{<<optional>>} pois ele é uma caracteristica opcional nos produtos gerados pela LPS.
    \end{column}
  
    \begin{column}{0.4\textwidth}
      \begin{figure}
      \pgfuseimage{fig:printer2SmartyBlocksPedacoOptional}
      \end{figure}
    \end{column}
    
  \end{columns}

\end{frame}


\begin{frame}
\frametitle{SyMPLES v1.0}

  \texttt{<<alternative\_OR>>}: estão sempre associadas aos pontos de variação. \textbf{Pelo menos uma} das variantes deverá ser escolhida para resolver o ponto de variação, ou seja, para estar presente na configuração da LPS.

\end{frame}


\begin{frame}
\frametitle{SyMPLES v1.0}
\framesubtitle{Exemplo de uso}

  \begin{columns}[onlytextwidth]
  
    \begin{column}{0.4\textwidth}
      \begin{figure}
      \pgfuseimage{fig:printer2SmartyBlocksPedacoAltOr}
      \end{figure}
    \end{column}
    
    \begin{column}{0.6\textwidth}
      \textbf{Exemplo}: o blocos \texttt{Usb2Connector}, \texttt{Usb3Connector} e \texttt{WifiConnector} foram marcados com o estereótipo \texttt{<<alternative\_OR>>} pois são características \textbf{alternativas inclusivas}, ou seja, pelo menos uma das variantes deve existir no produto gerado pela LPS.
    \end{column}
    
  \end{columns}

\end{frame}


\begin{frame}
\frametitle{SyMPLES v1.0}

  \texttt{<<alternative\_XOR>>}: estão sempre associadas aos pontos de variação. \textbf{Somente uma} das variantes deverá ser escolhida para resolver o ponto de variação.

\end{frame}


\begin{frame}
\frametitle{SyMPLES v1.0}
\framesubtitle{Exemplo de uso}

  \begin{columns}[onlytextwidth]
    
    \begin{column}{0.6\textwidth}
      \textbf{Exemplo}: o blocos \texttt{BWTonner} e \texttt{ColorTonner} foram marcados com o estereótipo \texttt{<<alternative\_XOR>>} pois são características \textbf{alternativas exclusivas}, ou seja, somente um bloco entre \texttt{BWTonner} e \texttt{ColorTonner} deve existir no produto gerado pela LPS.
    \end{column}
  
    \begin{column}{0.4\textwidth}
      \begin{figure}
      \pgfuseimage{fig:printer2SmartyBlocksPedacoAltXor}
      \end{figure}
    \end{column}
    
  \end{columns}

\end{frame}

\begin{frame}
\frametitle{SyMPLES v1.0}

  \texttt{<<variability>>}: indica uma variabilidade existente em um modelo UML.

\end{frame}


\begin{frame}
\frametitle{SyMPLES v1.0}
\framesubtitle{Exemplo de uso}

  \textbf{Exemplo}: o bloco \texttt{EmgPower} foi relacionado com uma nota estereotipada com \texttt{<<variability>>}, pois \texttt{EmgPower} representa um ponto de variação;

  \begin{figure}
  \pgfuseimage{fig:printer2SmartyBlocksPedacoVariability}
  \end{figure}

\end{frame}


\begin{frame}
\frametitle{SyMPLES v1.0}

  \texttt{<<requires>>}: indica um relacionamento de dependência (em \textit{UML}/\textit{SysML}) entre variantes no qual a
variante dependente (origem da dependência) só existirá em uma configuração se a variante relacionada (destino da dependência) existir.

\end{frame}


\begin{frame}
\frametitle{SyMPLES v1.0}
\framesubtitle{Exemplo de uso}
  
  \begin{columns}[onlytextwidth]
  
    \begin{column}{0.4\textwidth}
      \begin{figure}
      \pgfuseimage{fig:printer2SmartyBlocksPedacoRequires}
      \end{figure}
    \end{column}
    
    \begin{column}{0.6\textwidth}
      \textbf{Exemplo}: o bloco \texttt{ColorTonner} requer a existência do bloco \texttt{Usb3Connector}, então estes dois blocos foram relacionados com uma dependência estereotipada com \texttt{<<requires>>};
    \end{column}
    
  \end{columns}

\end{frame}


\begin{frame}
\frametitle{SyMPLES v1.0}

  \texttt{<<mutex>>}: indica um relacionamento de dependência (em \textit{UML}/\textit{SysML}) entre variantes no qual a variante dependente (origem da dependência) só existirá em uma configuração se a variante relacionada (destino da dependência) obrigatoriamente não existir.

  \begin{itemize}
    \item {
      São conhecidas como variantes mutuamente exclusivas.
    }
  \end{itemize}

\end{frame}


\begin{frame}
\frametitle{SyMPLES v1.0}
\framesubtitle{Exemplo de uso}
  
  \begin{columns}[onlytextwidth]
    
    \begin{column}{0.4\textwidth}
      \textbf{Exemplo}: o bloco \texttt{EmgPower} requer a exclusão do bloco \texttt{WifiConnector};
    \end{column}
  
    \begin{column}{0.6\textwidth}
      \begin{figure}
      \pgfuseimage{fig:printer2SmartyBlocksPedacoMutex}
      \end{figure}
    \end{column}
    
  \end{columns}

\end{frame}

\section{Treinamento - SyMPLES v1.0}

\begin{frame}
\frametitle{SyMPLES v1.0}
\framesubtitle{Treinamento}

  Com base na Descrição da LPS Mindstorms (\texttt{Doc 4.1}), e utilizando os Conceitos da Abordagem \textit{SyMPLES v1.0} (\texttt{Doc 3.2}), identifique as possíveis variabilidades contidas na LPS Mindstorms (\texttt{Doc 5.1}) de acordo com o seu julgamento.

\end{frame}

\begin{frame}
\frametitle{SyMPLES v1.0}
\framesubtitle{Treinamento}

  % \begin{figure}
  % \pgfuseimage{fig:roboMovelBlocks}
  % \end{figure}
  
  \href{run:./material/roboMovel-blocks.pdf}{\textcolor{blue}{$\gg$ Figura: LPS RoboMóvel}}

\end{frame}

\begin{frame}
\frametitle{SyMPLES v1.0}
\framesubtitle{Resultado}

  % \begin{figure}
  % \pgfuseimage{fig:roboMovelBlocksSMarty}
  % \end{figure}
  
  \href{run:./material/roboMovel-blocks-smarty.pdf}{\textcolor{blue}{$\gg$ Figura: LPS RoboMóvel com variabilidades representadas com \textit{SyMPLES v1.0}}}

\end{frame}
\section{Experimento - SyMPLES v1.0}

\begin{frame}[allowframebreaks]
\frametitle{SyMPLES v1.0}
\framesubtitle{Experimento}

  Com base nos conceitos de LP (\textit{Doc 3.1}), Conceitos da Abordagem \textit{SyMPLES v1.0} (\texttt{Doc 3.2}), na Descrição Resumida da LPS WeatherStation (\textit{Doc 4.3.1}) e na representação da LPS \textit{WeatherStation} contida no formulário do experimento (\textit{Doc 6.2}), responda as perguntas:
  
  \begin{enumerate}
    \item {
      É possível que um mesmo produto possua os blocos \texttt{Internet} e \texttt{External Sensors}?
    }
    \item {
      Um produto específico pode possuir 2 meios de saída de dados processados?
    }
    \item {
      A presença do dispositivo de alerta de tempestades requer a presença de um dispositivo local para a medição da velocidade do vento?
    }
    \item {
      Um produto específico pode fornecer seus dados via \texttt{WebServer} em formato \texttt{Texto Puro}?
    }
    \item {
      Um produto específico pode conter 0 (zero) blocos de alarmes?
    }
    \item {
      Qual a quantidade máxima de medidores locais que um produto específico pode conter?
    }
    \item {
      Um produto específico pode conter os dois idiomas definidos na LPS?
    }
    \item {
      Qual a quantidade de blocos no qual devem estar presentes em todos os produtos gerados pela LPS?
    }
    \item {
      Um produto gerado pela LPS \texttt{WeatherStation} dispensa a necessidade de um dispositivo para a obtenção de dados externos?
    }
    \item {
      Qual a quantidade máxima de blocos que um produto específico pode conter?
    }
  \end{enumerate}

\end{frame}

\begin{frame}
\frametitle{SyMPLES v1.0}
\framesubtitle{Exemplo de uso}

  \href{run:./material/est-met-blocks-inkscape.pdf}{\textcolor{blue}{$\gg$ Figura: LPS \textit{WeatherStation} com variabilidades representadas com a abordagem \textit{SyMPLES v1.0}}}

\end{frame}


% % SyMPLES v2.0
\section{Abordagem SyMPLES v2.0}

\begin{frame}
\frametitle{SyMPLES}

\begin{block}{SyMPLES}
   Abordagem de Desenvolvimento de Linha de Produto para Sistemas Embarcados baseada em \textit{SysML};
\end{block}

\end{frame}


\begin{frame}
\frametitle{SyMPLES}

\begin{itemize}
 \item {
   Combina modelos de alto nível e técnicas de LP para o desenvolvimento de Sistemas Embarcados;
 }
 \item {
   Utiliza a linguagem \textit{SysML} (extensão da linguagem \textit{UML}) como base para representação dos modelos LPS;
 }
 
\end{itemize}

\end{frame}

\begin{frame}
\frametitle{Linguagem CVL}

\begin{block}{CVL}
   CVL é uma linguagem para especificar e resolver variabilidades em linguagens baseada em \textit{MOF}, tais como \textit{UML} e \textit{SysML}.
\end{block}

\textit{SyMPLES v2.0} utiliza a linguagem CVL para o gerenciamento de variabilidades. A linguagem \textit{CVL} pode ser aplicada tanto para modelos \textit{UML} tanto para modelos \textit{SysML}.

\end{frame}

\begin{frame}
\frametitle{Linguagem CVL}

\begin{block}{Modelo base}
   Instância de um metamodelo \textit{MOF}, tal como modelos \textit{SysML};
\end{block}

\begin{block}{Modelo de variabilidade}
   É um conjunto de \textbf{pontos de variação}, especificações da variabilidade (\textbf{\textit{VSpecs}}) e \textbf{restrições} usadas para especificar variabilidades sobre o Modelo Base;
\end{block}

\end{frame}

\begin{frame}
\frametitle{Árvore VSpec (VSpec Tree)}

Os \textit{VSpecs} podem ser organizadas como árvores, onde a relação pai-filho organiza o espaço de resolução por imposição da estrutura e lógica nas resoluções permitidas.

Uma \textit{VSpec Tree} possui um nó raiz que representa o ponto de partida da materialização de um produto, e os outros nós serão responsáveis para representar a variabilidade da LPS.

\end{frame}

\begin{frame}
\frametitle{SyMPLES v2.0}
\framesubtitle{LPS de uma impressora - Diagrama de Definição de Blocos}

  \href{run:./material/printer2-blocks-cvl.pdf}{\textcolor{blue}{$\gg$ Figura: LPS \textit{Printer} com variabilidades representadas com a linguagem \textit{CVL}}}

\end{frame}

\begin{frame}
\frametitle{SyMPLES v2.0}
\framesubtitle{Modelo de variabilidades}

  O Modelo de Variabilidades é formado por:
  
  \begin{itemize}
    \item {
      Pontos de variação;
    }
    \item {
      Especificações de variabilidade (\textit{VSpec}) organizados em uma estrutura de árvore;
    }
    \item {
      Restrições entre os \textit{VSpecs};
    }
  \end{itemize}

\end{frame}

\begin{frame}
\frametitle{SyMPLES v2.0}
\framesubtitle{Identificação dos pontos de variação}

  Cada elemento no Modelo Base que \textbf{pode ou não existir} no Modelo Materializado deve ser representado por um \textbf{ponto de variação \textit{objectExistence}} no Modelo de Variabilidades;

\end{frame}

\begin{frame}
\frametitle{SyMPLES v2.0}
\framesubtitle{Definição dos VSpecs}

  Cada ponto de variação do tipo \textit{objectExistence} deve ser representado por um \textit{VSpec} do tipo escolha;

\end{frame}

\begin{frame}
\frametitle{SyMPLES v2.0}
\framesubtitle{Criação da Árvore VSpecs}

  \begin{itemize}
    \item {
      os \textit{VSpecs} do tipo escolha que são obrigatórios deverão ser ligados ao seu elemento pai por meio de uma linha sólida;
    }
    \item {
      os \textit{VSpecs} do tipo escolha que são opcionais deverão ser ligados ao seu elemento pai por meio de uma linha tracejada;
    }
    \item {
      os \textit{VSpecs} do tipo variável deverão estar relacionadas a \textit{VSpecs} do tipo \textit{choice}, e deverão ser ligados ao seu elemento pai por meio de uma linha sólida;
    }
  \end{itemize}
  
\end{frame}

\subsection{Exemplo}

\begin{frame}
\frametitle{SyMPLES v2.0}

  \textbf{Implicação negativa de resolução}: Uma resolução de escolha negativa requer uma resolução negativa para seus dependentes.

\end{frame}


\begin{frame}
\frametitle{SyMPLES v2.0}
\framesubtitle{Exemplo de uso}
  
  \begin{columns}[onlytextwidth]
  
    \begin{column}{0.5\textwidth}
      \begin{figure}
      \pgfuseimage{fig:printer2CvlBlocksIsImpliedByParentF}
      \end{figure}
    \end{column}
    
    \begin{column}{0.5\textwidth}
      \textbf{Exemplo}: o \textit{VSpec} \texttt{Tonner} se for resolvido negativamente, irá obrigar a resolução negativa para \texttt{BwTonner} e \texttt{ColorTonner};
    \end{column}
    
  \end{columns}

\end{frame}


\begin{frame}
\frametitle{SyMPLES v2.0}

  \textbf{Implicação positiva de resolução}: Cada \textit{VSpec} do tipo escolha possui um atributo \textit{isImpliedByParent} que, quando verdadeiro (\textit{true}), indica que se seu pai for resolvido positivamente, então ele deve ser decidido de forma positiva.

\end{frame}


\begin{frame}
\frametitle{SyMPLES v2.0}
\framesubtitle{Exemplo de uso}
  
  \begin{columns}[onlytextwidth]
    
    \begin{column}{0.6\textwidth}
      \textbf{Exemplo}: o \textit{VSpec} \texttt{Connector} possui \textit{isImpliedByParent=True}, simbolizado pela linha solida que interliga \texttt{Connector} e \texttt{Printer};
    \end{column}
  
    \begin{column}{0.4\textwidth}
      \begin{figure}
      \pgfuseimage{fig:printer2CvlBlocksIsImpliedByParentT}
      \end{figure}
    \end{column}
    
  \end{columns}

\end{frame}


\begin{frame}
\frametitle{SyMPLES v2.0}

  \textbf{Multiplicidade de grupo}: Um \textit{VSpec} pode conter uma multiplicidade de grupo, especificando o total de resoluções positivas que devem estar sob ele caso seja resolvido de forma positiva.
  
  \begin{itemize}
    \item {
      A multiplicidade de grupo é exibida por meio de um pequeno triângulo abaixo do \textit{VSpec}.
    }
  \end{itemize}

\end{frame}


\begin{frame}
\frametitle{SyMPLES v2.0}
\framesubtitle{Exemplo de uso}

  \begin{figure}
  \pgfuseimage{fig:printer2CvlBlocksMultiplicidadeGrupo}
  \end{figure}
  
  \textbf{Exemplo}: se o \textit{VSpec} \texttt{Connector} for resolvido positivamente, então no mínimo um \textit{VSpec} filho (de \texttt{Connector}) deverá ser resolvido positivamente;

\end{frame}

\begin{frame}
\frametitle{SyMPLES v2.0}

  \textbf{Restrição \textit{implies}}: Este tipo de restrição indica que a resolução positiva de um \textit{VSpec} requer a resolução positiva de outro \textit{VSpec};
  
  \textbf{Restrição \textit{implies not}}: Este tipo de restrição indica que a resolução positiva de um \textit{VSpec} requer a resolução negativa de outro \textit{VSpec};
  
  \begin{itemize}
    \item {
      As restrições do Modelo de Variabilidades são exibidas através de paralelogramos.
    }
  \end{itemize}

\end{frame}


\begin{frame}
\frametitle{SyMPLES v2.0}
\framesubtitle{Exemplo de uso}
  
  \textbf{Exemplo 1}: a resolução positiva de \texttt{ColorTonner} requer a resolução positiva de \texttt{Usb3Connector};
  
  \textbf{Exemplo 2}: a resolução positiva de \texttt{EmergencyPower} requer a resolução negativa de \texttt{WifiConnector};

  \begin{figure}
  \pgfuseimage{fig:printer2CvlBlocksImplies}
  \end{figure}

\end{frame}

\section{Treinamento - SyMPLES v2.0}

\begin{frame}
\frametitle{SyMPLES v2.0}
\framesubtitle{Treinamento}

  Com base na Descrição da LPS Mindstorms (\texttt{Doc 4.1}), e utilizando os Conceitos da Abordagem \textit{SyMPLES v2.0} (\texttt{Doc 3.3}), identifique as possíveis variabilidades contidas na LPS Mindstorms (\texttt{Doc 5.2}) de acordo com o seu julgamento.
  
\end{frame}

\begin{frame}
\frametitle{SyMPLES v2.0}
\framesubtitle{Treinamento}
  
  \href{run:./material/roboMovel-blocks.pdf}{\textcolor{blue}{$\gg$ Figura: LPS RoboMóvel}}

\end{frame}

\begin{frame}
\frametitle{SyMPLES v2.0}
\framesubtitle{Resultado}

  \href{run:./material/roboMovel-blocks-cvl.pdf}{\textcolor{blue}{$\gg$ Figura: LPS RoboMóvel com variabilidades representadas com \textit{SyMPLES v2.0}}}

\end{frame}
\section{Experimento - SyMPLES v2.0}

\begin{frame}[allowframebreaks]
\frametitle{SyMPLES v2.0}
\framesubtitle{Experimento}

  Com base nos conceitos de LP (\textit{Doc 3.1}), Conceitos da Abordagem \textit{SyMPLES v2.0} (\textit{Doc 3.3}), na Descrição Resumida da LPS WeatherStation (\textit{Doc 4.3.1}) e na representação da LPS \textit{WeatherStation} contida no formulário do experimento (\textit{Doc 6.3}), responda as perguntas:
  
  \begin{enumerate}
    \item {
      É possível que um mesmo produto possua os blocos \texttt{Internet} e \texttt{External Sensors}?
    }
    \item {
      Um produto específico pode possuir 2 meios de saída de dados processados?
    }
    \item {
      A presença do dispositivo de alerta de tempestades requer a presença de um dispositivo local para a medição da velocidade do vento?
    }
    \item {
      Um produto específico pode fornecer seus dados via \texttt{WebServer} em formato \texttt{Texto Puro}?
    }
    \item {
      Um produto específico pode conter 0 (zero) blocos de alarmes?
    }
    \item {
      Qual a quantidade máxima de medidores locais que um produto específico pode conter?
    }
    \item {
      Um produto específico pode conter os dois idiomas definidos na LPS?
    }
    \item {
      Qual a quantidade de blocos no qual devem estar presentes em todos os produtos gerados pela LPS?
    }
    \item {
      Um produto gerado pela LPS \texttt{WeatherStation} dispensa a necessidade de um dispositivo para a obtenção de dados externos?
    }
    \item {
      Qual a quantidade máxima de blocos que um produto específico pode conter?
    }
  \end{enumerate}

\end{frame}

\begin{frame}
\frametitle{SyMPLES v2.0}
\framesubtitle{Exemplo de uso}

  \href{run:./material/est-met-blocks-cvl.pdf}{\textcolor{blue}{$\gg$ Figura: LPS \textit{WeatherStation} com variabilidades representadas com a abordagem \textit{SyMPLES v2.0}}}

\end{frame}


\end{document}