\section{Experimento - SyMPLES v2.0}

\begin{frame}[allowframebreaks]
\frametitle{SyMPLES v2.0}
\framesubtitle{Experimento}

  Com base nos conceitos de LP (\textit{Doc 3.1}), Conceitos da Abordagem \textit{SyMPLES v2.0} (\textit{Doc 3.3}), na Descrição Resumida da LPS WeatherStation (\textit{Doc 4.3.1}) e na representação da LPS \textit{WeatherStation} contida no formulário do experimento (\textit{Doc 6.3}), responda as perguntas:
  
  \begin{enumerate}
    \item {
      É possível que um mesmo produto possua os blocos \texttt{Internet} e \texttt{External Sensors}?
    }
    \item {
      Um produto específico pode possuir 2 meios de saída de dados processados?
    }
    \item {
      A presença do dispositivo de alerta de tempestades requer a presença de um dispositivo local para a medição da velocidade do vento?
    }
    \item {
      Um produto específico pode fornecer seus dados via \texttt{WebServer} em formato \texttt{Texto Puro}?
    }
    \item {
      Um produto específico pode conter 0 (zero) blocos de alarmes?
    }
    \item {
      Qual a quantidade máxima de medidores locais que um produto específico pode conter?
    }
    \item {
      Um produto específico pode conter os dois idiomas definidos na LPS?
    }
    \item {
      Qual a quantidade de blocos no qual devem estar presentes em todos os produtos gerados pela LPS?
    }
    \item {
      Um produto gerado pela LPS \texttt{WeatherStation} dispensa a necessidade de um dispositivo para a obtenção de dados externos?
    }
    \item {
      Qual a quantidade máxima de blocos que um produto específico pode conter?
    }
  \end{enumerate}

\end{frame}

\begin{frame}
\frametitle{SyMPLES v2.0}
\framesubtitle{Exemplo de uso}

  \href{run:./material/est-met-blocks-cvl.pdf}{\textcolor{blue}{$\gg$ Figura: LPS \textit{WeatherStation} com variabilidades representadas com a abordagem \textit{SyMPLES v2.0}}}

\end{frame}
