\section{Abordagem SyMPLES v2.0}

\begin{frame}
\frametitle{SyMPLES}

\begin{block}{SyMPLES}
   Abordagem de Desenvolvimento de Linha de Produto para Sistemas Embarcados baseada em \textit{SysML};
\end{block}

\end{frame}


\begin{frame}
\frametitle{SyMPLES}

\begin{itemize}
 \item {
   Combina modelos de alto nível e técnicas de LP para o desenvolvimento de Sistemas Embarcados;
 }
 \item {
   Utiliza a linguagem \textit{SysML} (extensão da linguagem \textit{UML}) como base para representação dos modelos LPS;
 }
 
\end{itemize}

\end{frame}

\begin{frame}
\frametitle{Linguagem CVL}

\begin{block}{CVL}
   CVL é uma linguagem para especificar e resolver variabilidades em linguagens baseada em \textit{MOF}, tais como \textit{UML} e \textit{SysML}.
\end{block}

\textit{SyMPLES v2.0} utiliza a linguagem CVL para o gerenciamento de variabilidades. A linguagem \textit{CVL} pode ser aplicada tanto para modelos \textit{UML} tanto para modelos \textit{SysML}.

\end{frame}

\begin{frame}
\frametitle{Linguagem CVL}

\begin{block}{Modelo base}
   Instância de um metamodelo \textit{MOF}, tal como modelos \textit{SysML};
\end{block}

\begin{block}{Modelo de variabilidade}
   É um conjunto de \textbf{pontos de variação}, especificações da variabilidade (\textbf{\textit{VSpecs}}) e \textbf{restrições} usadas para especificar variabilidades sobre o Modelo Base;
\end{block}

\end{frame}

\begin{frame}
\frametitle{Árvore VSpec (VSpec Tree)}

Os \textit{VSpecs} podem ser organizadas como árvores, onde a relação pai-filho organiza o espaço de resolução por imposição da estrutura e lógica nas resoluções permitidas.

Uma \textit{VSpec Tree} possui um nó raiz que representa o ponto de partida da materialização de um produto, e os outros nós serão responsáveis para representar a variabilidade da LPS.

\end{frame}

\begin{frame}
\frametitle{SyMPLES v2.0}
\framesubtitle{LPS de uma impressora - Diagrama de Definição de Blocos}

  \href{run:./material/printer2-blocks-cvl.pdf}{\textcolor{blue}{$\gg$ Figura: LPS \textit{Printer} com variabilidades representadas com a linguagem \textit{CVL}}}

\end{frame}

\begin{frame}
\frametitle{SyMPLES v2.0}
\framesubtitle{Modelo de variabilidades}

  O Modelo de Variabilidades é formado por:
  
  \begin{itemize}
    \item {
      Pontos de variação;
    }
    \item {
      Especificações de variabilidade (\textit{VSpec}) organizados em uma estrutura de árvore;
    }
    \item {
      Restrições entre os \textit{VSpecs};
    }
  \end{itemize}

\end{frame}

\begin{frame}
\frametitle{SyMPLES v2.0}
\framesubtitle{Identificação dos pontos de variação}

  Cada elemento no Modelo Base que \textbf{pode ou não existir} no Modelo Materializado deve ser representado por um \textbf{ponto de variação \textit{objectExistence}} no Modelo de Variabilidades;

\end{frame}

\begin{frame}
\frametitle{SyMPLES v2.0}
\framesubtitle{Definição dos VSpecs}

  Cada ponto de variação do tipo \textit{objectExistence} deve ser representado por um \textit{VSpec} do tipo escolha;

\end{frame}

\begin{frame}
\frametitle{SyMPLES v2.0}
\framesubtitle{Criação da Árvore VSpecs}

  \begin{itemize}
    \item {
      os \textit{VSpecs} do tipo escolha que são obrigatórios deverão ser ligados ao seu elemento pai por meio de uma linha sólida;
    }
    \item {
      os \textit{VSpecs} do tipo escolha que são opcionais deverão ser ligados ao seu elemento pai por meio de uma linha tracejada;
    }
    \item {
      os \textit{VSpecs} do tipo variável deverão estar relacionadas a \textit{VSpecs} do tipo \textit{choice}, e deverão ser ligados ao seu elemento pai por meio de uma linha sólida;
    }
  \end{itemize}
  
\end{frame}

\subsection{Exemplo}

\begin{frame}
\frametitle{SyMPLES v2.0}

  \textbf{Implicação negativa de resolução}: Uma resolução de escolha negativa requer uma resolução negativa para seus dependentes.

\end{frame}


\begin{frame}
\frametitle{SyMPLES v2.0}
\framesubtitle{Exemplo de uso}
  
  \begin{columns}[onlytextwidth]
  
    \begin{column}{0.5\textwidth}
      \begin{figure}
      \pgfuseimage{fig:printer2CvlBlocksIsImpliedByParentF}
      \end{figure}
    \end{column}
    
    \begin{column}{0.5\textwidth}
      \textbf{Exemplo}: o \textit{VSpec} \texttt{Tonner} se for resolvido negativamente, irá obrigar a resolução negativa para \texttt{BwTonner} e \texttt{ColorTonner};
    \end{column}
    
  \end{columns}

\end{frame}


\begin{frame}
\frametitle{SyMPLES v2.0}

  \textbf{Implicação positiva de resolução}: Cada \textit{VSpec} do tipo escolha possui um atributo \textit{isImpliedByParent} que, quando verdadeiro (\textit{true}), indica que se seu pai for resolvido positivamente, então ele deve ser decidido de forma positiva.

\end{frame}


\begin{frame}
\frametitle{SyMPLES v2.0}
\framesubtitle{Exemplo de uso}
  
  \begin{columns}[onlytextwidth]
    
    \begin{column}{0.6\textwidth}
      \textbf{Exemplo}: o \textit{VSpec} \texttt{Connector} possui \textit{isImpliedByParent=True}, simbolizado pela linha solida que interliga \texttt{Connector} e \texttt{Printer};
    \end{column}
  
    \begin{column}{0.4\textwidth}
      \begin{figure}
      \pgfuseimage{fig:printer2CvlBlocksIsImpliedByParentT}
      \end{figure}
    \end{column}
    
  \end{columns}

\end{frame}


\begin{frame}
\frametitle{SyMPLES v2.0}

  \textbf{Multiplicidade de grupo}: Um \textit{VSpec} pode conter uma multiplicidade de grupo, especificando o total de resoluções positivas que devem estar sob ele caso seja resolvido de forma positiva.
  
  \begin{itemize}
    \item {
      A multiplicidade de grupo é exibida por meio de um pequeno triângulo abaixo do \textit{VSpec}.
    }
  \end{itemize}

\end{frame}


\begin{frame}
\frametitle{SyMPLES v2.0}
\framesubtitle{Exemplo de uso}

  \begin{figure}
  \pgfuseimage{fig:printer2CvlBlocksMultiplicidadeGrupo}
  \end{figure}
  
  \textbf{Exemplo}: se o \textit{VSpec} \texttt{Connector} for resolvido positivamente, então no mínimo um \textit{VSpec} filho (de \texttt{Connector}) deverá ser resolvido positivamente;

\end{frame}

\begin{frame}
\frametitle{SyMPLES v2.0}

  \textbf{Restrição \textit{implies}}: Este tipo de restrição indica que a resolução positiva de um \textit{VSpec} requer a resolução positiva de outro \textit{VSpec};
  
  \textbf{Restrição \textit{implies not}}: Este tipo de restrição indica que a resolução positiva de um \textit{VSpec} requer a resolução negativa de outro \textit{VSpec};
  
  \begin{itemize}
    \item {
      As restrições do Modelo de Variabilidades são exibidas através de paralelogramos.
    }
  \end{itemize}

\end{frame}


\begin{frame}
\frametitle{SyMPLES v2.0}
\framesubtitle{Exemplo de uso}
  
  \textbf{Exemplo 1}: a resolução positiva de \texttt{ColorTonner} requer a resolução positiva de \texttt{Usb3Connector};
  
  \textbf{Exemplo 2}: a resolução positiva de \texttt{EmergencyPower} requer a resolução negativa de \texttt{WifiConnector};

  \begin{figure}
  \pgfuseimage{fig:printer2CvlBlocksImplies}
  \end{figure}

\end{frame}
